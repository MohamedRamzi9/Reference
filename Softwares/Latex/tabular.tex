\usepackage{tabularx}

% Table
\begin{table}[h] % start of table environment, with placement option 'here', can also be t (top), b (bottom), p (page of floats), ! (override), H (exactly here, requires float package)
    
    \begin{tabular}{| % begin of tabular environment with various column types, | for vertical lines between columns and around the table, optional
            c| % centered column, vertically aligned in the top
            l| % left aligned column, vertically aligned in the top
            r| % right aligned column, vertically aligned in the top 
            p{3cm}| % fixed width column, top aligned, justified text
            m{3cm}| % fixed width column, middle aligned, justified text
            b{3cm}| % fixed width column, bottom aligned, justified text
            >{\centering\arraybackslash}p{3cm}| % fixed width column, centered text, not justified
            >{\raggedright\arraybackslash}p{3cm}| % fixed width column, left aligned text, not justified
            >{\raggedleft\arraybackslash}p{3cm}| % fixed width column, right aligned text, not justified
        }
        \centering % center align the table, \raggedright for right, or \raggedleft for left
        \hline % horizontal line on top
        Header 1 & Header 2 & Header 3 \\ % table headers, & separates columns, \\ ends the row
        \hline % horizontal line
        Row 1 Col 1 & Row 1 Col 2 & Row 1 Col 3 \\ % first row
        Row 2 Col 1 & Row 2 Col 2 & Row 2 Col 3 \\ % second row
        \hline % horizontal line at the bottom
    \end{tabular}
    
    \begin{tabularx}{\textwidth}{|X|X|X|} % start of tabularx environment with specified width (here \textwidth), X columns auto-adjust width to fit their content
    \end{tabularx} % end of tabularx environment

    \caption{Basic Table} % caption for the table
    \label{tab:basic_table} % label for referencing the table
\end{table} % end of table environment
